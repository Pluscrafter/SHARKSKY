\hypertarget{stm32f7xx__hal__sd_8c}{}\section{Drivers/\+S\+T\+M32\+F7xx\+\_\+\+H\+A\+L\+\_\+\+Driver/\+Src/stm32f7xx\+\_\+hal\+\_\+sd.c File Reference}
\label{stm32f7xx__hal__sd_8c}\index{Drivers/STM32F7xx\_HAL\_Driver/Src/stm32f7xx\_hal\_sd.c@{Drivers/STM32F7xx\_HAL\_Driver/Src/stm32f7xx\_hal\_sd.c}}


SD card H\+AL module driver. This file provides firmware functions to manage the following functionalities of the Secure Digital (SD) peripheral\+:  


{\ttfamily \#include \char`\"{}stm32f7xx\+\_\+hal.\+h\char`\"{}}\newline


\subsection{Detailed Description}
SD card H\+AL module driver. This file provides firmware functions to manage the following functionalities of the Secure Digital (SD) peripheral\+: 

\begin{DoxyAuthor}{Author}
M\+CD Application Team
\begin{DoxyItemize}
\item Initialization and de-\/initialization functions
\item IO operation functions
\item Peripheral Control functions
\item SD card Control functions
\end{DoxyItemize}
\end{DoxyAuthor}
\begin{DoxyVerb}==============================================================================
                      ##### How to use this driver #####
==============================================================================
[..]
  This driver implements a high level communication layer for read and write from/to 
  this memory. The needed STM32 hardware resources (SDMMC and GPIO) are performed by 
  the user in HAL_SD_MspInit() function (MSP layer).                             
  Basically, the MSP layer configuration should be the same as we provide in the 
  examples.
  You can easily tailor this configuration according to hardware resources.

[..]
  This driver is a generic layered driver for SDMMC memories which uses the HAL 
  SDMMC driver functions to interface with SD and uSD cards devices. 
  It is used as follows:

  (#)Initialize the SDMMC low level resources by implement the HAL_SD_MspInit() API:
      (##) Enable the SDMMC interface clock using __HAL_RCC_SDMMC_CLK_ENABLE(); 
      (##) SDMMC pins configuration for SD card
          (+++) Enable the clock for the SDMMC GPIOs using the functions __HAL_RCC_GPIOx_CLK_ENABLE();   
          (+++) Configure these SDMMC pins as alternate function pull-up using HAL_GPIO_Init()
                and according to your pin assignment;
      (##) DMA Configuration if you need to use DMA process (HAL_SD_ReadBlocks_DMA()
           and HAL_SD_WriteBlocks_DMA() APIs).
          (+++) Enable the DMAx interface clock using __HAL_RCC_DMAx_CLK_ENABLE(); 
          (+++) Configure the DMA using the function HAL_DMA_Init() with predeclared and filled. 
      (##) NVIC configuration if you need to use interrupt process when using DMA transfer.
          (+++) Configure the SDMMC and DMA interrupt priorities using functions
                HAL_NVIC_SetPriority(); DMA priority is superior to SDMMC's priority
          (+++) Enable the NVIC DMA and SDMMC IRQs using function HAL_NVIC_EnableIRQ()
          (+++) SDMMC interrupts are managed using the macros __HAL_SD_ENABLE_IT() 
                and __HAL_SD_DISABLE_IT() inside the communication process.
          (+++) SDMMC interrupts pending bits are managed using the macros __HAL_SD_GET_IT()
                and __HAL_SD_CLEAR_IT()
      (##) NVIC configuration if you need to use interrupt process (HAL_SD_ReadBlocks_IT()
           and HAL_SD_WriteBlocks_IT() APIs).
          (+++) Configure the SDMMC interrupt priorities using function
                HAL_NVIC_SetPriority();
          (+++) Enable the NVIC SDMMC IRQs using function HAL_NVIC_EnableIRQ()
          (+++) SDMMC interrupts are managed using the macros __HAL_SD_ENABLE_IT() 
                and __HAL_SD_DISABLE_IT() inside the communication process.
          (+++) SDMMC interrupts pending bits are managed using the macros __HAL_SD_GET_IT()
                and __HAL_SD_CLEAR_IT()
  (#) At this stage, you can perform SD read/write/erase operations after SD card initialization  

       
*** SD Card Initialization and configuration ***
================================================    
[..]
  To initialize the SD Card, use the HAL_SD_Init() function. It Initializes 
  SDMMC IP (STM32 side) and the SD Card, and put it into StandBy State (Ready for data transfer). 
  This function provide the following operations:

  (#) Initialize the SDMMC peripheral interface with defaullt configuration.
      The initialization process is done at 400KHz. You can change or adapt 
      this frequency by adjusting the "ClockDiv" field. 
      The SD Card frequency (SDMMC_CK) is computed as follows:

         SDMMC_CK = SDMMCCLK / (ClockDiv + 2)

      In initialization mode and according to the SD Card standard, 
      make sure that the SDMMC_CK frequency doesn't exceed 400KHz.

      This phase of initialization is done through SDMMC_Init() and 
      SDMMC_PowerState_ON() SDMMC low level APIs.

  (#) Initialize the SD card. The API used is HAL_SD_InitCard().
      This phase allows the card initialization and identification 
      and check the SD Card type (Standard Capacity or High Capacity)
      The initialization flow is compatible with SD standard.

      This API (HAL_SD_InitCard()) could be used also to reinitialize the card in case 
      of plug-off plug-in.

  (#) Configure the SD Card Data transfer frequency. By Default, the card transfer 
      frequency is set to 24MHz. You can change or adapt this frequency by adjusting 
      the "ClockDiv" field.
      In transfer mode and according to the SD Card standard, make sure that the 
      SDMMC_CK frequency doesn't exceed 25MHz and 50MHz in High-speed mode switch.
      To be able to use a frequency higher than 24MHz, you should use the SDMMC 
      peripheral in bypass mode. Refer to the corresponding reference manual 
      for more details.

  (#) Select the corresponding SD Card according to the address read with the step 2.
  
  (#) Configure the SD Card in wide bus mode: 4-bits data.

*** SD Card Read operation ***
==============================
[..] 
  (+) You can read from SD card in polling mode by using function HAL_SD_ReadBlocks(). 
      This function allows the read of 512 bytes blocks.
      You can choose either one block read operation or multiple block read operation 
      by adjusting the "NumberOfBlocks" parameter.
      After this, you have to ensure that the transfer is done correctly. The check is done
      through HAL_SD_GetCardState() function for SD card state.

  (+) You can read from SD card in DMA mode by using function HAL_SD_ReadBlocks_DMA().
      This function allows the read of 512 bytes blocks.
      You can choose either one block read operation or multiple block read operation 
      by adjusting the "NumberOfBlocks" parameter.
      After this, you have to ensure that the transfer is done correctly. The check is done
      through HAL_SD_GetCardState() function for SD card state.
      You could also check the DMA transfer process through the SD Rx interrupt event.

  (+) You can read from SD card in Interrupt mode by using function HAL_SD_ReadBlocks_IT().
      This function allows the read of 512 bytes blocks.
      You can choose either one block read operation or multiple block read operation 
      by adjusting the "NumberOfBlocks" parameter.
      After this, you have to ensure that the transfer is done correctly. The check is done
      through HAL_SD_GetCardState() function for SD card state.
      You could also check the IT transfer process through the SD Rx interrupt event.

*** SD Card Write operation ***
=============================== 
[..] 
  (+) You can write to SD card in polling mode by using function HAL_SD_WriteBlocks(). 
      This function allows the read of 512 bytes blocks.
      You can choose either one block read operation or multiple block read operation 
      by adjusting the "NumberOfBlocks" parameter.
      After this, you have to ensure that the transfer is done correctly. The check is done
      through HAL_SD_GetCardState() function for SD card state.

  (+) You can write to SD card in DMA mode by using function HAL_SD_WriteBlocks_DMA().
      This function allows the read of 512 bytes blocks.
      You can choose either one block read operation or multiple block read operation 
      by adjusting the "NumberOfBlocks" parameter.
      After this, you have to ensure that the transfer is done correctly. The check is done
      through HAL_SD_GetCardState() function for SD card state.
      You could also check the DMA transfer process through the SD Tx interrupt event.  

  (+) You can write to SD card in Interrupt mode by using function HAL_SD_WriteBlocks_IT().
      This function allows the read of 512 bytes blocks.
      You can choose either one block read operation or multiple block read operation 
      by adjusting the "NumberOfBlocks" parameter.
      After this, you have to ensure that the transfer is done correctly. The check is done
      through HAL_SD_GetCardState() function for SD card state.
      You could also check the IT transfer process through the SD Tx interrupt event.

*** SD card status ***
====================== 
[..]
  (+) The SD Status contains status bits that are related to the SD Memory 
      Card proprietary features. To get SD card status use the HAL_SD_GetCardStatus().

*** SD card information ***
=========================== 
[..]
  (+) To get SD card information, you can use the function HAL_SD_GetCardInfo().
      It returns useful information about the SD card such as block size, card type,
      block number ...

*** SD card CSD register ***
============================
[..]
  (+) The HAL_SD_GetCardCSD() API allows to get the parameters of the CSD register.
      Some of the CSD parameters are useful for card initialization and identification.

*** SD card CID register ***
============================
[..]
  (+) The HAL_SD_GetCardCID() API allows to get the parameters of the CID register.
      Some of the CSD parameters are useful for card initialization and identification.

*** SD HAL driver macros list ***
==================================
[..]
  Below the list of most used macros in SD HAL driver.
     
  (+) __HAL_SD_ENABLE : Enable the SD device
  (+) __HAL_SD_DISABLE : Disable the SD device
  (+) __HAL_SD_DMA_ENABLE: Enable the SDMMC DMA transfer
  (+) __HAL_SD_DMA_DISABLE: Disable the SDMMC DMA transfer
  (+) __HAL_SD_ENABLE_IT: Enable the SD device interrupt
  (+) __HAL_SD_DISABLE_IT: Disable the SD device interrupt
  (+) __HAL_SD_GET_FLAG:Check whether the specified SD flag is set or not
  (+) __HAL_SD_CLEAR_FLAG: Clear the SD's pending flags

[..]
  (@) You can refer to the SD HAL driver header file for more useful macros 
    
*** Callback registration ***
=============================================
[..]
  The compilation define USE_HAL_SD_REGISTER_CALLBACKS when set to 1
  allows the user to configure dynamically the driver callbacks.

  Use Functions @ref HAL_SD_RegisterCallback() to register a user callback,
  it allows to register following callbacks:
    (+) TxCpltCallback : callback when a transmission transfer is completed.
    (+) RxCpltCallback : callback when a reception transfer is completed.
    (+) ErrorCallback : callback when error occurs.
    (+) AbortCpltCallback : callback when abort is completed.
    (+) MspInitCallback    : SD MspInit.
    (+) MspDeInitCallback  : SD MspDeInit.
  This function takes as parameters the HAL peripheral handle, the Callback ID
  and a pointer to the user callback function.

  Use function @ref HAL_SD_UnRegisterCallback() to reset a callback to the default
  weak (surcharged) function. It allows to reset following callbacks:
    (+) TxCpltCallback : callback when a transmission transfer is completed.
    (+) RxCpltCallback : callback when a reception transfer is completed.
    (+) ErrorCallback : callback when error occurs.
    (+) AbortCpltCallback : callback when abort is completed.
    (+) MspInitCallback    : SD MspInit.
    (+) MspDeInitCallback  : SD MspDeInit.
  This function) takes as parameters the HAL peripheral handle and the Callback ID.

  By default, after the @ref HAL_SD_Init and if the state is HAL_SD_STATE_RESET
  all callbacks are reset to the corresponding legacy weak (surcharged) functions.
  Exception done for MspInit and MspDeInit callbacks that are respectively
  reset to the legacy weak (surcharged) functions in the @ref HAL_SD_Init 
  and @ref  HAL_SD_DeInit only when these callbacks are null (not registered beforehand).
  If not, MspInit or MspDeInit are not null, the @ref HAL_SD_Init and @ref HAL_SD_DeInit
  keep and use the user MspInit/MspDeInit callbacks (registered beforehand)

  Callbacks can be registered/unregistered in READY state only.
  Exception done for MspInit/MspDeInit callbacks that can be registered/unregistered
  in READY or RESET state, thus registered (user) MspInit/DeInit callbacks can be used
  during the Init/DeInit.
  In that case first register the MspInit/MspDeInit user callbacks
  using @ref HAL_SD_RegisterCallback before calling @ref HAL_SD_DeInit 
  or @ref HAL_SD_Init function.

  When The compilation define USE_HAL_SD_REGISTER_CALLBACKS is set to 0 or
  not defined, the callback registering feature is not available 
  and weak (surcharged) callbacks are used.\end{DoxyVerb}


\begin{DoxyAttention}{Attention}

\end{DoxyAttention}
\subsubsection*{\begin{center}\copyright{} Copyright (c) 2017 S\+T\+Microelectronics. All rights reserved.\end{center} }

This software component is licensed by ST under B\+SD 3-\/Clause license, the \char`\"{}\+License\char`\"{}; You may not use this file except in compliance with the License. You may obtain a copy of the License at\+: opensource.\+org/licenses/\+B\+S\+D-\/3-\/\+Clause 