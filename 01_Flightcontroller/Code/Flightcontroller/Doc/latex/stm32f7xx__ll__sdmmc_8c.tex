\hypertarget{stm32f7xx__ll__sdmmc_8c}{}\section{Drivers/\+S\+T\+M32\+F7xx\+\_\+\+H\+A\+L\+\_\+\+Driver/\+Src/stm32f7xx\+\_\+ll\+\_\+sdmmc.c File Reference}
\label{stm32f7xx__ll__sdmmc_8c}\index{Drivers/STM32F7xx\_HAL\_Driver/Src/stm32f7xx\_ll\_sdmmc.c@{Drivers/STM32F7xx\_HAL\_Driver/Src/stm32f7xx\_ll\_sdmmc.c}}


S\+D\+M\+MC Low Layer H\+AL module driver.  


{\ttfamily \#include \char`\"{}stm32f7xx\+\_\+hal.\+h\char`\"{}}\newline


\subsection{Detailed Description}
S\+D\+M\+MC Low Layer H\+AL module driver. 

\begin{DoxyAuthor}{Author}
M\+CD Application Team \begin{DoxyVerb}     This file provides firmware functions to manage the following 
     functionalities of the SDMMC peripheral:
      + Initialization/de-initialization functions
      + I/O operation functions
      + Peripheral Control functions 
      + Peripheral State functions
\end{DoxyVerb}

\end{DoxyAuthor}
\begin{DoxyVerb}==============================================================================
                     ##### SDMMC peripheral features #####
==============================================================================        
  [..] The SD/SDMMC MMC card host interface (SDMMC) provides an interface between the APB2
       peripheral bus and MultiMedia cards (MMCs), SD memory cards, SDMMC cards and CE-ATA
       devices.
  
  [..] The SDMMC features include the following:
       (+) Full compliance with MultiMedia Card System Specification Version 4.2. Card support
           for three different databus modes: 1-bit (default), 4-bit and 8-bit
       (+) Full compatibility with previous versions of MultiMedia Cards (forward compatibility)
       (+) Full compliance with SD Memory Card Specifications Version 2.0
       (+) Full compliance with SD I/O Card Specification Version 2.0: card support for two
           different data bus modes: 1-bit (default) and 4-bit
       (+) Full support of the CE-ATA features (full compliance with CE-ATA digital protocol
           Rev1.1)
       (+) Data transfer up to 48 MHz for the 8 bit mode
       (+) Data and command output enable signals to control external bidirectional drivers.
               
 
                         ##### How to use this driver #####
==============================================================================
  [..]
    This driver is a considered as a driver of service for external devices drivers 
    that interfaces with the SDMMC peripheral.
    According to the device used (SD card/ MMC card / SDMMC card ...), a set of APIs 
    is used in the device's driver to perform SDMMC operations and functionalities.
 
    This driver is almost transparent for the final user, it is only used to implement other
    functionalities of the external device.
 
  [..]
    (+) The SDMMC clock (SDMMCCLK = 48 MHz) is coming from a specific output of PLL 
        (PLL48CLK). Before start working with SDMMC peripheral make sure that the
        PLL is well configured.
        The SDMMC peripheral uses two clock signals:
        (++) SDMMC adapter clock (SDMMCCLK = 48 MHz)
        (++) APB2 bus clock (PCLK2)
     
        -@@- PCLK2 and SDMMC_CK clock frequencies must respect the following condition:
             Frequency(PCLK2) >= (3 / 8 x Frequency(SDMMC_CK))

    (+) Enable/Disable peripheral clock using RCC peripheral macros related to SDMMC
        peripheral.

    (+) Enable the Power ON State using the SDMMC_PowerState_ON(SDMMCx) 
        function and disable it using the function SDMMC_PowerState_OFF(SDMMCx).
              
    (+) Enable/Disable the clock using the __SDMMC_ENABLE()/__SDMMC_DISABLE() macros.

    (+) Enable/Disable the peripheral interrupts using the macros __SDMMC_ENABLE_IT(hSDMMC, IT) 
        and __SDMMC_DISABLE_IT(hSDMMC, IT) if you need to use interrupt mode. 

    (+) When using the DMA mode 
        (++) Configure the DMA in the MSP layer of the external device
        (++) Active the needed channel Request 
        (++) Enable the DMA using __SDMMC_DMA_ENABLE() macro or Disable it using the macro
             __SDMMC_DMA_DISABLE().

    (+) To control the CPSM (Command Path State Machine) and send 
        commands to the card use the SDMMC_SendCommand(SDMMCx), 
        SDMMC_GetCommandResponse() and SDMMC_GetResponse() functions. First, user has
        to fill the command structure (pointer to SDMMC_CmdInitTypeDef) according 
        to the selected command to be sent.
        The parameters that should be filled are:
         (++) Command Argument
         (++) Command Index
         (++) Command Response type
         (++) Command Wait
         (++) CPSM Status (Enable or Disable).

        -@@- To check if the command is well received, read the SDMMC_CMDRESP
            register using the SDMMC_GetCommandResponse().
            The SDMMC responses registers (SDMMC_RESP1 to SDMMC_RESP2), use the
            SDMMC_GetResponse() function.

    (+) To control the DPSM (Data Path State Machine) and send/receive 
         data to/from the card use the SDMMC_DataConfig(), SDMMC_GetDataCounter(), 
        SDMMC_ReadFIFO(), SDMMC_WriteFIFO() and SDMMC_GetFIFOCount() functions.

  *** Read Operations ***
  =======================
  [..]
    (#) First, user has to fill the data structure (pointer to
        SDMMC_DataInitTypeDef) according to the selected data type to be received.
        The parameters that should be filled are:
         (++) Data TimeOut
         (++) Data Length
         (++) Data Block size
         (++) Data Transfer direction: should be from card (To SDMMC)
         (++) Data Transfer mode
         (++) DPSM Status (Enable or Disable)
                                   
    (#) Configure the SDMMC resources to receive the data from the card
        according to selected transfer mode (Refer to Step 8, 9 and 10).

    (#) Send the selected Read command (refer to step 11).
                  
    (#) Use the SDMMC flags/interrupts to check the transfer status.

  *** Write Operations ***
  ========================
  [..]
   (#) First, user has to fill the data structure (pointer to
       SDMMC_DataInitTypeDef) according to the selected data type to be received.
       The parameters that should be filled are:
        (++) Data TimeOut
        (++) Data Length
        (++) Data Block size
        (++) Data Transfer direction:  should be to card (To CARD)
        (++) Data Transfer mode
        (++) DPSM Status (Enable or Disable)

   (#) Configure the SDMMC resources to send the data to the card according to 
       selected transfer mode.
                   
   (#) Send the selected Write command.
                  
   (#) Use the SDMMC flags/interrupts to check the transfer status.
     
  *** Command management operations ***
  =====================================
  [..]
   (#) The commands used for Read/Write//Erase operations are managed in 
       separate functions. 
       Each function allows to send the needed command with the related argument,
       then check the response.
       By the same approach, you could implement a command and check the response.\end{DoxyVerb}


\begin{DoxyAttention}{Attention}

\end{DoxyAttention}
\subsubsection*{\begin{center}\copyright{} Copyright (c) 2017 S\+T\+Microelectronics. All rights reserved.\end{center} }

This software component is licensed by ST under B\+SD 3-\/Clause license, the \char`\"{}\+License\char`\"{}; You may not use this file except in compliance with the License. You may obtain a copy of the License at\+: opensource.\+org/licenses/\+B\+S\+D-\/3-\/\+Clause 