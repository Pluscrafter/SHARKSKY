\hypertarget{stm32f7xx__hal__uart_8c}{}\section{Drivers/\+S\+T\+M32\+F7xx\+\_\+\+H\+A\+L\+\_\+\+Driver/\+Src/stm32f7xx\+\_\+hal\+\_\+uart.c File Reference}
\label{stm32f7xx__hal__uart_8c}\index{Drivers/STM32F7xx\_HAL\_Driver/Src/stm32f7xx\_hal\_uart.c@{Drivers/STM32F7xx\_HAL\_Driver/Src/stm32f7xx\_hal\_uart.c}}


U\+A\+RT H\+AL module driver. This file provides firmware functions to manage the following functionalities of the Universal Asynchronous Receiver Transmitter Peripheral (U\+A\+RT).  


{\ttfamily \#include \char`\"{}stm32f7xx\+\_\+hal.\+h\char`\"{}}\newline


\subsection{Detailed Description}
U\+A\+RT H\+AL module driver. This file provides firmware functions to manage the following functionalities of the Universal Asynchronous Receiver Transmitter Peripheral (U\+A\+RT). 

\begin{DoxyAuthor}{Author}
M\+CD Application Team
\begin{DoxyItemize}
\item Initialization and de-\/initialization functions
\item IO operation functions
\item Peripheral Control functions
\end{DoxyItemize}
\end{DoxyAuthor}
\begin{DoxyVerb}===============================================================================
                       ##### How to use this driver #####
===============================================================================
 [..]
   The UART HAL driver can be used as follows:

   (#) Declare a UART_HandleTypeDef handle structure (eg. UART_HandleTypeDef huart).
   (#) Initialize the UART low level resources by implementing the HAL_UART_MspInit() API:
       (++) Enable the USARTx interface clock.
       (++) UART pins configuration:
           (+++) Enable the clock for the UART GPIOs.
           (+++) Configure these UART pins as alternate function pull-up.
       (++) NVIC configuration if you need to use interrupt process (HAL_UART_Transmit_IT()
            and HAL_UART_Receive_IT() APIs):
           (+++) Configure the USARTx interrupt priority.
           (+++) Enable the NVIC USART IRQ handle.
       (++) UART interrupts handling:
             -@@-  The specific UART interrupts (Transmission complete interrupt,
               RXNE interrupt, RX/TX FIFOs related interrupts and Error Interrupts)
               are managed using the macros __HAL_UART_ENABLE_IT() and __HAL_UART_DISABLE_IT()
               inside the transmit and receive processes.
       (++) DMA Configuration if you need to use DMA process (HAL_UART_Transmit_DMA()
            and HAL_UART_Receive_DMA() APIs):
           (+++) Declare a DMA handle structure for the Tx/Rx channel.
           (+++) Enable the DMAx interface clock.
           (+++) Configure the declared DMA handle structure with the required Tx/Rx parameters.
           (+++) Configure the DMA Tx/Rx channel.
           (+++) Associate the initialized DMA handle to the UART DMA Tx/Rx handle.
           (+++) Configure the priority and enable the NVIC for the transfer complete interrupt on the DMA Tx/Rx channel.

   (#) Program the Baud Rate, Word Length, Stop Bit, Parity, Hardware
       flow control and Mode (Receiver/Transmitter) in the huart handle Init structure.

   (#) If required, program UART advanced features (TX/RX pins swap, auto Baud rate detection,...)
       in the huart handle AdvancedInit structure.

   (#) For the UART asynchronous mode, initialize the UART registers by calling
       the HAL_UART_Init() API.

   (#) For the UART Half duplex mode, initialize the UART registers by calling
       the HAL_HalfDuplex_Init() API.

   (#) For the UART LIN (Local Interconnection Network) mode, initialize the UART registers
       by calling the HAL_LIN_Init() API.

   (#) For the UART Multiprocessor mode, initialize the UART registers
       by calling the HAL_MultiProcessor_Init() API.

   (#) For the UART RS485 Driver Enabled mode, initialize the UART registers
       by calling the HAL_RS485Ex_Init() API.

   [..]
   (@) These API's (HAL_UART_Init(), HAL_HalfDuplex_Init(), HAL_LIN_Init(), HAL_MultiProcessor_Init(),
       also configure the low level Hardware GPIO, CLOCK, CORTEX...etc) by
       calling the customized HAL_UART_MspInit() API.

   ##### Callback registration #####
   ==================================

   [..]
   The compilation define USE_HAL_UART_REGISTER_CALLBACKS when set to 1
   allows the user to configure dynamically the driver callbacks.

   [..]
   Use Function @ref HAL_UART_RegisterCallback() to register a user callback.
   Function @ref HAL_UART_RegisterCallback() allows to register following callbacks:
   (+) TxHalfCpltCallback        : Tx Half Complete Callback.
   (+) TxCpltCallback            : Tx Complete Callback.
   (+) RxHalfCpltCallback        : Rx Half Complete Callback.
   (+) RxCpltCallback            : Rx Complete Callback.
   (+) ErrorCallback             : Error Callback.
   (+) AbortCpltCallback         : Abort Complete Callback.
   (+) AbortTransmitCpltCallback : Abort Transmit Complete Callback.
   (+) AbortReceiveCpltCallback  : Abort Receive Complete Callback.
   (+) WakeupCallback            : Wakeup Callback.
   (+) RxFifoFullCallback        : Rx Fifo Full Callback.
   (+) TxFifoEmptyCallback       : Tx Fifo Empty Callback.
   (+) MspInitCallback           : UART MspInit.
   (+) MspDeInitCallback         : UART MspDeInit.
   This function takes as parameters the HAL peripheral handle, the Callback ID
   and a pointer to the user callback function.

   [..]
   Use function @ref HAL_UART_UnRegisterCallback() to reset a callback to the default
   weak (surcharged) function.
   @ref HAL_UART_UnRegisterCallback() takes as parameters the HAL peripheral handle,
   and the Callback ID.
   This function allows to reset following callbacks:
   (+) TxHalfCpltCallback        : Tx Half Complete Callback.
   (+) TxCpltCallback            : Tx Complete Callback.
   (+) RxHalfCpltCallback        : Rx Half Complete Callback.
   (+) RxCpltCallback            : Rx Complete Callback.
   (+) ErrorCallback             : Error Callback.
   (+) AbortCpltCallback         : Abort Complete Callback.
   (+) AbortTransmitCpltCallback : Abort Transmit Complete Callback.
   (+) AbortReceiveCpltCallback  : Abort Receive Complete Callback.
   (+) WakeupCallback            : Wakeup Callback.
   (+) RxFifoFullCallback        : Rx Fifo Full Callback.
   (+) TxFifoEmptyCallback       : Tx Fifo Empty Callback.
   (+) MspInitCallback           : UART MspInit.
   (+) MspDeInitCallback         : UART MspDeInit.

   [..]
   By default, after the @ref HAL_UART_Init() and when the state is HAL_UART_STATE_RESET
   all callbacks are set to the corresponding weak (surcharged) functions:
   examples @ref HAL_UART_TxCpltCallback(), @ref HAL_UART_RxHalfCpltCallback().
   Exception done for MspInit and MspDeInit functions that are respectively
   reset to the legacy weak (surcharged) functions in the @ref HAL_UART_Init()
   and @ref HAL_UART_DeInit() only when these callbacks are null (not registered beforehand).
   If not, MspInit or MspDeInit are not null, the @ref HAL_UART_Init() and @ref HAL_UART_DeInit()
   keep and use the user MspInit/MspDeInit callbacks (registered beforehand).

   [..]
   Callbacks can be registered/unregistered in HAL_UART_STATE_READY state only.
   Exception done MspInit/MspDeInit that can be registered/unregistered
   in HAL_UART_STATE_READY or HAL_UART_STATE_RESET state, thus registered (user)
   MspInit/DeInit callbacks can be used during the Init/DeInit.
   In that case first register the MspInit/MspDeInit user callbacks
   using @ref HAL_UART_RegisterCallback() before calling @ref HAL_UART_DeInit()
   or @ref HAL_UART_Init() function.

   [..]
   When The compilation define USE_HAL_UART_REGISTER_CALLBACKS is set to 0 or
   not defined, the callback registration feature is not available
   and weak (surcharged) callbacks are used.\end{DoxyVerb}


\begin{DoxyAttention}{Attention}

\end{DoxyAttention}
\subsubsection*{\begin{center}\copyright{} Copyright (c) 2017 S\+T\+Microelectronics. All rights reserved.\end{center} }

This software component is licensed by ST under B\+SD 3-\/Clause license, the \char`\"{}\+License\char`\"{}; You may not use this file except in compliance with the License. You may obtain a copy of the License at\+: opensource.\+org/licenses/\+B\+S\+D-\/3-\/\+Clause 