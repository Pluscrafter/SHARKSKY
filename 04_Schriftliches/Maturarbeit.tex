\documentclass[12pt,a4paper, ngerman]{article}
\usepackage{babel}
\usepackage[round]{natbib}
\begin{document}
\bibliographystyle{alphadin}
\title{\large Maturitätsarbeit an der Kantonsschule Zürich Nord \\ \Huge Regelungstechnik \\ \huge PID-Parameter anhand einem Quadrocopter erforschen}
\date{\today}
\author{Charpoan Kong \\ M5d \\ Kantonsschule Zürich Nord}
\maketitle
\pagenumbering{gobble}

\newpage
\clearpage
\pagenumbering{gobble}
\tableofcontents
\newpage
\pagenumbering{arabic}

\section{Einleitung}

\newpage
\section{Was ist ein Quadrocopter?}
Ein Quadrocopter ist ein Flugobjekt mit 4 Propellern. 

Geschichte ... 

In der heutigen Zeit finden Quadrocopter viele Anwendungen, wie z.B. im Militär, bei Suchaktionen oder auch als Spielzeug.

Mit dieser Arbeit will ich mit dem Bau und dessen Regelung befasssen. Der Quadrocopter wird selbst gebau mit dem Hauptziel den Flightcontroller selbst zu designen, bauen und programmieren. So sollte die Regelung selbst implementiert werden. Dannach sollten Versuche gemacht werden, die bestimmen sollten, wie der Regler stabiler gemacht werden kann.

\section{PID-Regelung und Filter}
\subsection{PID-Reglelung}
Eine Regelung ist ein Steuerung mit Rückkoppelung. Es wird ein Wert z.B. die Drehzahl eines Motors überwacht und je nach gewünschter Drehzahl, die Spannung des Motors geregelt, dass er auch diesen halten kann, auch wenn eine Last am Motor hängt.    

\subsection{Filter 1}

\section{Äusserer Aufbau}
\subsection{Motoren}
\subsection{Akku}

\section{Der Flightcontroller}

\subsection{Genereller Aufbau}

\subsubsection{Microcontroller}

\subsubsection{Sensoren}

\subsubsection{Empfänger}

\subsubsection{Schnitstellen}


\subsection{Leiterplatte}

\subsubsection{Design}

\subsubsection{Bau}

\subsection{Programmierung}

\subsubsection{Motoren und Timing}

\subsubsection{PID Implementierung}

\section{Fernbedienung}

\section{Versuche}

\subsection{PID absoluter Winkel oder Winkelgeschwindigkeit}

\subsubsection{Absoluter Winkel}

\subsubsection{Winkelgeschwindigkeit}

\subsubsection{Ergebnisse}

\subsection{Filter}

\subsubsection{Ohne Filter}

\subsubsection{Filter 1}

\subsubsection{Filter 2}

\subsubsection{Ergebnisse}

\section{Schlussfolgerung}




\newpage
\section{Glossar}


\newpage
\section{Quellenverzeichnis}
\subsection{Literaturverzeichnis}
\subsection{Abbildungsverzeichnis}
\subsection{Personenverzeichnis}

\end{document}
